\section{सारांश तालिका}\label{sec:case-summary-table}
\begin{tabularx}{\linewidth}{ c X c X }
    \caption{कारक}\label{tab:case-summary-table}\tabularnewline
    \toprule

    \midrule
    \textbf{कारक} & \textbf{प्रश्न} & \textbf{\ru{падеж}} & \textbf{\ru{вопрос}} \tabularnewline
    \midrule
    \endfirsthead

    \midrule
    \textbf{कारक} & \textbf{प्रश्न} & \textbf{\ru{падеж}} & \textbf{\ru{вопрос}} \tabularnewline
    \midrule
    \endhead

    \midrule
    \multicolumn{4}{r}{\footnotesize{अगले पृष्ट पर जारी}}
    \endfoot

    \bottomrule
    \multicolumn{4}{r}{\footnotesize{इति तालिका~\ref{tab:case-summary-table} }} \tabularnewline
    \endlastfoot

    कर्ता & किसने  & \ru{именительный} & \ru{кто?} \par \ru{что?}
    \tabularnewline

    कर्म & किसको  & \ru{винительный} & \ru{кого?} \par \ru{что?}
    \tabularnewline

    संबंध & किसका \par किसके \par किसकी & \ru{Предложный} & \ru{о ком?} \par \ru{о чём?}
    \tabularnewline

    संप्रदान & किसके लिए  & \ru{Дательный} & \ru{кому?} \par \ru{чему?}
    \tabularnewline

    अधिकरण & में, पर  & \ru{Родительный} & \ru{кого?} \par \ru{чего?}
    \tabularnewline

    करण & किससे (किसके द्वारा) & \ru{Творительный} & \ru{кем?} \par \ru{чем?} \par \ru{за кем?} \par \ru{за чем?}
    \tabularnewline

    अपादान & किससे (अलगाव) & \ru{Творительный} & \ru{с кем?} \par \ru{с чем?} \par \ru{от кем?} \par \ru{от чем?}
\end{tabularx}

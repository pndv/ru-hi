\chapter{संज्ञा}\label{ch: noun}\index{संज्ञा}
इसे रूसी में \ru{существительное} कहते हैं।

\section{लिंग}\label{sec:noun-gender}
हिन्दी कि तरह रूसी भाषा में भी संज्ञा के तीन लिंग [\ru{род}] \index{\ru{род}} होते हैं :

स्त्रीलिंग\index{स्त्रीलिंग}  \ru{Женский род}\index{\ru{род}!\ru{Женский}} [ज़ेन्सकीय रोद]
पुलिंग\index{पुलिंग} \ru{Мужской род}\index{\ru{род}!\ru{Мужской}} [मूज़कोय रोद]
नपुंसकलिंग\index{नपुंसकलिंग} \ru{Средний род}\index{\ru{род}!\ru{Средний}} [स्रेदनीय रोद]

सामान्यत: स्त्रीलिंग शब्द \ru{-а} (आ), \ru{-ь} (\ru{мякий знак}),  अथवा \ru{-я} (या)  से अंत होते हैं, नपुंसकलिंग प्राय: \ru{-о} (ओ) अथवा
\ru{-е} (ये) से अंत होते हैं, पुलिंग शब्द बाकी किसी भी अक्षर से अंत हो सकते हैं। यह ध्यान रखिए कि इस नियम के बहुत अपवाद हैं, ऐसे अपवादों का उल्लेख
आवश्यकतानुसार किया जाएगा। रूसी भाषा ने कई भाषाओं के शब्दों को आत्मसात किया है, ऐसे शब्दों का लिंग-निर्धारण निम्नलिखित नियमों से होता है:
\begin{enumerate}
    \item या तो रूसी भाषा के जिस शब्द को वह विदेशी शब्द दर्शित कर रहा हो, उसके अनुसार
    \begin{itemize}
        \item tornado (चक्रवात) अंग्रेज़ी शब्द का संबंध हवा () से है, इसलिए इसे पुलिंग माना गया है
        \item (सलामी) (एक् प्रकार का मांसाहारी खाद्य)
    \end{itemize}
    \item  यदि उस शब्द का कोई रूसी समकक्ष न हो तो, उनकी मूल भाषा के अनुरूप किया जाता है, जैसे
    \begin{itemize}
        \item कोफे - कॉफी (पुलिंग)
    \end{itemize}
\end{enumerate}

% TODO:
% ~\cite{franke2012}
% а + й = я
% ь + а = я
% й + у = ю
\section{शब्दों के रूप (प्रथम प्रकार)}\label{sec:noun-endings-first-declension}
रूसी में सजीव और निर्जीव वस्तुओं के लिए अलग रूप होते हैं। हाँलाकि दोनों रूपों में अधिकतर समानतायें हैं, परंतु कुछ कारकों के लिए दोनों, जैसे कर्म कारक, में यह
रूप भिन्न हो जाते हैं~\cite{readyruss2021}।

\subsection{\sru{р}-कारांत शब्दों का रूप (निर्जीव वस्तु)}\label{subsec:noun-endings-first-declension-hard-inanimate-male}
\gencasetable{
    \sru{р}-कारांत संज्ञा;
    tab:noun-endings-first-declension-hard-inanimate-male;
    театр;
    театры;
    театр;
    театры;
    театра;
    театров;
    театре;
    театрых;
    театру;
    театрам;
    театром;
    театрами
}

\subsection{\sru{к}-कारांत शब्दों का रूप (सजीव वस्तु)}\label{subsec:noun-endings-first-declension-hard-animate-male}
\gencasetable {
    \sru{к}-कारांत संज्ञा;
    tab:noun-endings-first-declension-hard-animate-male;
    ма́льчик;
    ма́льчикы;
    ма́льчика;
    ма́льчиков;
    ма́льчика;
    ма́льчиков;
    ма́льчике;
    ма́льчиках;
    ма́льчику;
    ма́льчикам;
    ма́льчиком;
    ма́льчиками
}

\section{शब्दों के रूप (द्वितीय प्रकार)}\label{sec:noun-endings-second-declension}
यह प्रकार अधिकतर स्त्रीलिंग के लिए प्रयोग होता है ~\cite{readyruss2021, levine2009}, परंतु कुछ शब्द जैसे \ru{папа} (पिता), \ru{дядя} (चाचा, ताऊ,
मामा) पुलिंग होते हैं।

\subsection{स्त्रीलिंग \sru{а}-कारांत शब्दों का रूप (निर्जीव वस्तु)}\label{subsec:noun-endings-second-declension-hard}
\ru{газе́та} = अखबार/समाचारपत्र
\gencasetable {
    \sru{а}-कारांत संज्ञा;
    tab:noun-endings-first-declension-hard;
    газе́та;
    газе́ты;
    газе́ту;
    газе́ты;
    газе́ты;
    газе́т;
    газе́те;
    газе́тах;
    газе́те;
    газе́там;
    газе́той;
    газе́тами
}

इसी प्रकार: \ru{де́душка} (दादा - पुलिंग), \ru{ма́ма} (माता - स्त्रीलिंग),  \ru{па́па} (पिता - पुलिंग), \ru{маши́на} (गाड़ी/car - स्त्रीलिंग)

\subsection{\sru{я}-कारांत शब्दों का रूप}\label{subsec:noun-endings-second-declension-soft}
\ru{ку́хня} = रसोई

\gencasetable {
    \sru{я}-कारांत संज्ञा;
    tab:noun-endings-second-declension-soft;
    ку́хня;
    ку́хни;
    ку́хню;
    ку́хни;
    ку́хни;
    ку́хонь;
    ку́хне;
    ку́хнях;
    ку́хне;
    ку́хням;
    ку́хней;
    ку́хнями
}

इसी प्रकार: \ru{дядя} (चाचा/ताऊ/मामा - पुलिंग)

\section{शब्दों के रूप (तृतीय प्रकार)}\label{sec:noun-endings-third-declension-female}
यह प्रकार स्त्रीलिंग शब्दों के लिए प्रयोग होता है, जिनका अंत \ru{ь (мякий знак)} से होता हो ~\cite{readyruss2021,levine2009}।

\subsection{स्त्रीलिंग \sru{ь}-कारांत शब्दों का रूप (निर्जीव वस्तु)}\label{subsec:noun-endings-second-declension-hard-inanimate-female}
\ru{жизнь} = जीवन
\gencasetable {
    \sru{ь}-कारांत संज्ञा;
    tab:noun-endings-third-declension-female;
    жизнь;
    жи́зни;
    жизнь;
    жи́зни;
    жи́зни;
    жи́зней;
    жи́зни;
    жи́знях;
    жи́зни;
    жи́зням;
    жи́знью;
    жи́знями
}

इसी प्रकार: \ru{лошадь} (अश्व), \ru{плошадь} (वर्ग, square), \ru{кровь} (रक्त), \ru{ночь} (रात्रि)

\section{बहुवचनीय तथा एकवचनीय संज्ञा}\label{sec: noun-singular-plural-form-only}
कुछ संज्ञाओं के केवल एकवचनीय या बहुवचनीय रूप ही होते हैं। इनका विवरण यहाँ दिया गया है।

\section{बहुवचनीय संज्ञा}\label{sec: noun-plural-form-only}\index{संज्ञा!बहुवचनीय}

\section{एकवचनीय संज्ञा}\label{sec: noun-singular-form-only}\index{संज्ञा!एकवचनीय}


\section{शब्दों के रूप (द्वितीय प्रकार)}\label{sec:noun-endings-second-declension}
यह प्रकार अधिकतर स्त्रीलिंग के लिए प्रयोग होता है ~\cite{readyruss2021}।

\subsection{स्त्रीलिंग \sru{а}-कारांत शब्दों का रूप (निर्जीव वस्तु)}\label{subsec:noun-endings-second-declension-hard-inanimate-female}
\gencasetable {
    \sru{а}-कारांत संज्ञा;
    tab:noun-endings-first-declension-hard-inanimate-female;
    газе́та;
    газе́ты;
    газе́ту;
    газе́ты;
    газе́ты;
    газе́т;
    газе́те;
    газе́тах;
    газе́те;
    газе́там;
    газе́той;
    газе́тами
}

\subsection{स्त्रीलिंग \sru{а}-कारांत शब्दों का रूप (सजीव वस्तु)}\label{subsec:noun-endings-second-declension-hard-animate-female}
\gencasetable {
    \sru{а}-कारांत संज्ञा;
    tab:noun-endings-second-declension-hard-animate-female;
    ма́ма;
    ма́мы;
    ма́му;
    мам;
    ма́мы;
    мам;
    ма́ме;
    мама́х;
    ма́ме;
    ма́мам;
    ма́мой;
    ма́мами
}

\subsection{स्त्रीलिंग \sru{я}-कारांत शब्दों का रूप (निर्जीव वस्तु)}\label{subsec:noun-endings-second-declension-soft-inanimate-female}
\gencasetable {
    \sru{я}-कारांत संज्ञा;
    tab:noun-endings-second-declension-soft-inanimate-female;
    ку́хня;
    ку́хни;
    ку́хню;
    ку́хни;
    ку́хни;
    ку́хонь;
    ку́хне;
    ку́хнях;
    ку́хне;
    ку́хням;
    ку́хней;
    ку́хнями
}

\subsection{स्त्रीलिंग \sru{ия}-कारांत शब्दों का रूप}\label{subsec:noun-endings-second-declension-soft-iya-female}
\gencasetable {
    \sru{ия}-कारांत संज्ञा;
    tab:noun-endings-second-declension-soft-iya-female;
    па́ртия;
    па́ртии;
    па́ртию;
    па́ртии;
    па́ртии;
    па́ртий;
    па́ртии;
    па́ртиях;
    па́ртии;
    па́ртиям;
    па́ртией;
    па́ртиями
}

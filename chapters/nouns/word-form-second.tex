\section{शब्दों के रूप (द्वितीय प्रकार)}\label{sec:noun-endings-second-declension}
यह प्रकार अधिकतर स्त्रीलिंग के लिए प्रयोग होता है ~\cite{readyruss2021, levine2009}, परंतु कुछ शब्द जैसे \ru{папа} (पिता), \ru{дядя} (चाचा, ताऊ,
मामा) पुलिंग होते हैं।

\subsection{स्त्रीलिंग \sru{а}-कारांत शब्दों का रूप (निर्जीव वस्तु)}\label{subsec:noun-endings-second-declension-hard}
\ru{газе́та} = अखबार/समाचारपत्र
\gencasetable {
    \sru{а}-कारांत संज्ञा;
    tab:noun-endings-first-declension-hard;
    газе́та;
    газе́ты;
    газе́ту;
    газе́ты;
    газе́ты;
    газе́т;
    газе́те;
    газе́тах;
    газе́те;
    газе́там;
    газе́той;
    газе́тами
}

इसी प्रकार: \ru{де́душка} (दादा - पुलिंग), \ru{ма́ма} (माता - स्त्रीलिंग),  \ru{па́па} (पिता - पुलिंग), \ru{маши́на} (गाड़ी/car - स्त्रीलिंग)

\subsection{\sru{я}-कारांत शब्दों का रूप}\label{subsec:noun-endings-second-declension-soft}
\ru{ку́хня} = रसोई

\gencasetable {
    \sru{я}-कारांत संज्ञा;
    tab:noun-endings-second-declension-soft;
    ку́хня;
    ку́хни;
    ку́хню;
    ку́хни;
    ку́хни;
    ку́хонь;
    ку́хне;
    ку́хнях;
    ку́хне;
    ку́хням;
    ку́хней;
    ку́хнями
}

इसी प्रकार: \ru{дядя} (चाचा/ताऊ/मामा - पुलिंग)

\section{शब्दों के रूप (प्रथम प्रकार)}\label{sec:noun-endings-first-declension}
रूसी में सजीव और निर्जीव वस्तुओं के लिए अलग रूप होते हैं। हाँलाकि दोनों रूपों में अधिकतर समानतायें हैं, परंतु कुछ कारकों के लिए दोनों, जैसे कर्म कारक, में यह
रूप भिन्न हो जाते हैं~\cite{readyruss2021}।

\subsection{कठोर-अंत पुलिंग शब्दों का रूप (निर्जीव वस्तु)}\label{subsec:noun-endings-first-declension-hard-inanimate-male}
\gencasetable{
    \sru{р}-कारांत संज्ञा \par \sru{теáтр} = रंगमंच, थिएटर (theatre);
    tab:noun-endings-first-declension-hard-inanimate-male;
    теáтр;
    теáтры;
    теáтр;
    теáтры;
    теáтра;
    теáтров;
    теáтре;
    теáтрых;
    теáтру;
    теáтрам;
    теáтром;
    теáтрами
}

\subsection{कठोर-अंत पुलिंग शब्दों का रूप (सजीव वस्तु)}\label{subsec:noun-endings-first-declension-hard-animate-male}
\ru{ма́льчик} = बालक

\gencasetable {
    \sru{к}-कारांत संज्ञा;
    tab:noun-endings-first-declension-hard-animate-male;
    ма́льчик;
    ма́льчикы;
    ма́льчика;
    ма́льчиков;
    ма́льчика;
    ма́льчиков;
    ма́льчике;
    ма́льчиках;
    ма́льчику;
    ма́льчикам;
    ма́льчиком;
    ма́льчиками
}


\subsection{कोमल-अंत \sru{й} पुलिंग शब्दों का रूप (निर्जीव वस्तु)}\label{subsec:noun-endings-first-declension-y-inanimate-male}
\ru{музе́й} = म्‍यूजि़यम
\gencasetable {
    \sru{й}-कारांत संज्ञा;
    tab:noun-endings-first-declension-y-inanimate-male;
    музе́й;
    музеи;
    музей;
    музеи;
    музее;
    музеях;
    музею;
    музеям;
    музея;
    музеев;
    музеем;
    музеями
}

\subsection{कोमल-अंत \sru{ь (мякий знак)} पुलिंग शब्दों का रूप (निर्जीव वस्तु)}\label{subsec:noun-endings-first-declension-b-inanimate-male}
\ru{портфе́ль} = ब्रीफ़केस / अटैची
\gencasetable {
    \sru{ь}-कारांत संज्ञा;
    tab:noun-endings-first-declension-b-inanimate-male;
    портфе́ль; %Именительный
    портфе́ли;
    портфе́ль; %винительный
    портфе́ли;
    портфе́ле; %Предложный
    портфе́лях;
    портфе́лю; %Дательный
    портфе́лям;
    портфе́ля; %Родительный
    портфе́лей;
    портфе́лем; %Творительный
    портфе́лями
}


\subsection{\sru{o}--कारांत नपुंसकलिंग शब्दों का रूप}\label{subsec:noun-endings-first-declension-o-neuter}
\ru{кре́сло} = कुर्सी
\gencasetable {
    \sru{o}-कारांत संज्ञा;
    tab:noun-endings-first-declension-o-neuter;
    кре́сло; %Именительный
    кре́сла;
    кре́сло; %винительный
    кре́сла;
    кре́сле; %Предложный
    кре́слах;
    кре́слу; %Дательный
    кре́слам;
    кре́сла; %Родительный
    кре́сел;
    кре́слом; %Творительный
    кре́слами
}
इसी प्रकार: \ru{колено} (घुटना), \ru{облако} (मेघ, बादल), \ru{яблоко} (सेब), \ru{болóто} (दलदल)



\subsection{\sru{е}--कारांत नपुंसकलिंग शब्दों का रूप}\label{subsec:noun-endings-first-declension-e-neuter}
\ru{здáние} = बिल्डिंग
\gencasetable {
    \sru{е}-कारांत संज्ञा;
    tab:noun-endings-first-declension-e-neuter;
    здáние; %Именительный
    здáния;
    здáние; %винительный
    здáния;
    здáнии; %Предложный
    здáниях;
    здáнию; %Дательный
    здáниям;
    здáния; %Родительный
    здáний;
    здáнием; %Творительный
    здáниями
}
इसी प्रकार: \ru{Жильё}, \ru{шитьё}, \ru{нытьё}, \ru{питьё}, \ru{остриё}, \ru{ружьё}, \ru{копьё}, \ru{жульё}, \ru{гнильё},
\ru{враньё}, \ru{старьё}, \ru{сырьё}, \ru{чутьё}, \ru{бритьё}



\subsection{\sru{ё}--कारांत नपुंसकलिंग शब्दों का रूप}\label{subsec:noun-endings-first-declension-yo-neuter}
\ru{бельё} = अंडरवियर (underwear)
\gencasetable {
    \sru{ё}-कारांत संज्ञा;
    tab:noun-endings-first-declension-yo-neuter;
    бельё; %Именительный
    Белья;
    бельё; %винительный
    Белья;
    белье; %Предложный
    Бельях;
    белью; %Дательный
    Бельям;
    белья; %Родительный
    Белей;
    бельём; %Творительный
    Бельями
}
इसी प्रकार: \ru{Жильё}, \ru{шитьё}, \ru{нытьё}, \ru{питьё}, \ru{остриё}, \ru{ружьё}, \ru{копьё}, \ru{жульё}, \ru{гнильё},
\ru{враньё}, \ru{старьё}, \ru{сырьё}, \ru{чутьё}, \ru{бритьё}

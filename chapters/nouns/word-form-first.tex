\section{शब्दों के रूप (प्रथम प्रकार)}\label{sec:noun-endings-first-declension}
रूसी में सजीव और निर्जीव वस्तुओं के लिए अलग रूप होते हैं। हाँलाकि दोनों रूपों में अधिकतर समानतायें हैं, परंतु कुछ कारकों के लिए दोनों, जैसे कर्म कारक, में यह
रूप भिन्न हो जाते हैं~\cite{readyruss2021}।

\subsection{\sru{р}-कारांत शब्दों का रूप (निर्जीव वस्तु)}\label{subsec:noun-endings-first-declension-hard-inanimate-male}
\gencasetable{
    \sru{р}-कारांत संज्ञा;
    tab:noun-endings-first-declension-hard-inanimate-male;
    театр;
    театры;
    театр;
    театры;
    театра;
    театров;
    театре;
    театрых;
    театру;
    театрам;
    театром;
    театрами
}

\subsection{\sru{к}-कारांत शब्दों का रूप (सजीव वस्तु)}\label{subsec:noun-endings-first-declension-hard-animate-male}
\gencasetable {
    \sru{к}-कारांत संज्ञा;
    tab:noun-endings-first-declension-hard-animate-male;
    ма́льчик;
    ма́льчикы;
    ма́льчика;
    ма́льчиков;
    ма́льчика;
    ма́льчиков;
    ма́льчике;
    ма́льчиках;
    ма́льчику;
    ма́льчикам;
    ма́льчиком;
    ма́льчиками
}

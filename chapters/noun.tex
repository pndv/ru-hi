\chapter{संज्ञा}\label{ch: noun}\index{संज्ञा}


\section{लिंग}\label{sec:noun-gender}
हिन्दी कि तरह रूसी भाषा में भी संज्ञा के तीन लिंग [\ru{род}] \index{\ru{род}} होते हैं :

स्त्रीलिंग\index{स्त्रीलिंग}  \ru{Женский род}\index{\ru{род}!\ru{Женский}} [ज़ेन्सकीय रोद]
पुलिंग\index{पुलिंग} \ru{Мужской род}\index{\ru{род}!\ru{Мужской}} [मूज़कोय रोद]
नपुंसकलिंग\index{नपुंसकलिंग} \ru{Средний род}\index{\ru{род}!\ru{Средний}} [स्रेदनीय रोद]

सामान्यत: स्त्रीलिंग शब्द \ru{-а} (आ), \ru{-ь} (\ru{мякий знак}),  अथवा \ru{-я} (या)  से अंत होते हैं, नपुंसकलिंग प्राय: \ru{-о} (ओ) अथवा
\ru{-е} (ये) से अंत होते हैं, पुलिंग शब्द बाकी किसी भी अक्षर से अंत हो सकते हैं। यह ध्यान रखिए कि इस नियम के बहुत अपवाद हैं, ऐसे अपवादों का उल्लेख
आवश्यकतानुसार किया जाएगा। रूसी भाषा ने कई भाषाओं के शब्दों को आत्मसात किया है, ऐसे शब्दों का लिंग-निर्धारण निम्नलिखित नियमों से होता है:
\begin{enumerate}
    \item या तो रूसी भाषा के जिस शब्द को वह विदेशी शब्द दर्शित कर रहा हो, उसके अनुसार
    \begin{itemize}
        \item tornado (चक्रवात) अंग्रेज़ी शब्द का संबंध हवा () से है, इसलिए इसे पुलिंग माना गया है
        \item (सलामी) (एक् प्रकार का मांसाहारी खाद्य)
    \end{itemize}
    \item  यदि उस शब्द का कोई रूसी समकक्ष न हो तो, उनकी मूल भाषा के अनुरूप किया जाता है, जैसे
    \begin{itemize}
        \item कोफे - कॉफी (पुलिंग)
    \end{itemize}
\end{enumerate}


\section{शब्दों के रूप}\label{sec:noun-endings}

\subsection{\sru{р}-कारांत शब्दों का रूप}\label{subsec:noun-endings-hard}
\gencasetable {\sru{р}-कारांत संज्ञा;tab:noun-endings-hard;театр;театры;театр;театры;театра;театров;театре;театрых;театру;театрам;театром;театрами}

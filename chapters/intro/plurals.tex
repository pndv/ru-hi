\section{वचन}\label{sec:intro-plurals}\index{वचन}
  रूसी में वचन के दो रूप होते हैं: एकवचन और बहुवचन। यद्यपि हिन्दी तथा संस्कृत-भाषियों को तीसरे रूप, द्विवचन, का भान होगा, परंतु उसका इसका उपयोग रूसी में नहीं होता।
  बहुवचन का प्रयोग रूसी में, हिन्दी कि तरह दो तरीके से होता है: पहला, एक से ज्यादा वस्तुओं/व्यक्तियों इत्यादि को दर्शाने में, और दूसरा, किसी को मानपूर्वक संबोधित करने में।
  उदाहरण के लिए:
  \begin{itemize}
    \item  \ru{что ты делаешь?}\par तुम क्या कर रहे हो?
    \item  \ru{что вы делаете?}\par आप क्या कर रहे हैं ?
  \end{itemize}


  रूसी परिवेश सामान्यतय: बहुवचन \ru{вы}\index{\ru{вы}} (आप) का प्रयोग इन परिस्थितियों में होता है:
  \begin{itemize}
    \item लोगों/भीड़ को संबोधित करने के लिए
    \item अपने से ओहदे में बड़े लोगों को (परिवार के बाहर)
    \item अध्यापकगणों के लिए
    \item कार्यक्षेत्र में अपने से ऊंचे पदाधिकारियों के लिए
    \item कार्यक्षेत्र में अपने सहकर्मचारियों के लिए (जिनसे आप ज़्यादा करीब न हों)
  \end{itemize}

  रूसी परिवेश सामान्यतय: एकवचन \ru{ты}\index{\ru{ты}} (तू | तुम) का प्रयोग इन परिस्थितियों में होता है:
  \begin{itemize}
    \item दोस्तों के मध्यस्त
    \item पशुओं और अपने पालतू जानवरों के लिए
    \item  परिवार के लोगों के बीच में, इसमें उनकी उम्र या पद से कोई सरोकार नहीं होता। यह आत्मीयता का लक्षण है। लोग अपने माता, पिता,
    दादा, दादी, चाचा इत्यादि को हमेशा तुम से संबोधित करते हैं। यह हिन्दी-भाषियों के लिए थोड़ा अजीब जान पड़ता है क्योंकि हमें परिवार मे अपने से बड़ों को हमेशा आदरपूर्वक `आप'
    से बुलाना सिखाया गया है।
  \end{itemize}
  यह ध्यान रखिए कि हिन्दी में हम कई बार तुम का प्रयोग द्विवचन या बहुवचन में भी कर लेते हैं, जैसे कि "तुम दोनों क्या कर हो?" या "तुम लोगों को एक साथ पढ़ाया जाएगा";
  ऐसा रूसी भाषा में वर्जित है, \ru{ты} और सभी एकवचनीय शब्दों का प्रयोग हमेशा एक वस्तु/व्यक्ति के लिए होता है।

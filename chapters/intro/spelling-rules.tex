\section{वर्तनी के नियम (Spelling Rules)~\cite{levine_2009, robin_2012}}\label{sec:alpha-spelling}
  रूसी भाषा के अक्षरों को लिखते समय निम्नलिखित नियम ध्यान रखिए, इनका विवरण आवश्यकतानुसार आगे भी दिया जाएगा। इन नियमों
  को स्मरित रखना अत्यधिक आवश्यक है, क्योंकि यह नियम आगे कारक~\ref{ch:cases} में प्रयोग होंगे, जो हिन्दी व्याकरण कि
  तरह, रूसी भाषा में महत्व रखते हैं।

  \subsection{\ru{К}, \ru{Г}, \ru{Х}}\label{subsec:alpha-spelling-k-g-h}
    \begin{enumerate}
      \item इनके आगे हमेशा \ru{-у} (ऊ) लगेगा, कभी भी \ru{-ю} (यू) नहीं
      \item इनके आगे हमेशा \ru{-ы} लगेगा, कभी भी \ru{-и} (ई) नहीं
      \item इनके आगे हमेशा \ru{-а} (आ) लगेगा, कभी भी \ru{-я} (या) नहीं
    \end{enumerate}

  \subsection{\ru{Ц}}\label{subsec:alpha-spelling-ts}
    \begin{enumerate}
      \item इनके आगे हमेशा \ru{-у} (ऊ) लगेगा, कभी भी \ru{-ю} (यू) नहीं
      \item इनके आगे हमेशा \ru{-ы} लगेगा, कभी भी \ru{-и} (ई) नहीं
      \item इसके आगे हमेशा सबल \ru{-\'е} (ये) लगेगा, कभी भी साधारण \ru{-о} (ओ) नहीं
    \end{enumerate}

  \subsection{\ru{Ж}, \ru{Ч}, \ru{Ш}, \ru{Щ}}\label{subsec:alpha-spelling-zh-ch-sh-scsh-ts}
    \begin{enumerate}
      \item इनके आगे हमेशा \ru{-у} (ऊ) लगेगा, कभी भी \ru{-ю} (यू) नहीं
      \item इनके आगे हमेशा \ru{-ы} लगेगा, कभी भी \ru{-и} (ई) नहीं
      \item इनके आगे हमेशा \ru{-а} (आ) लगेगा, कभी भी \ru{-я} (या) नहीं
      \item इनके आगे हमेशा सबल \ru{-\'е} (ये) लगेगा, कभी भी साधारण \ru{-о} (ओ) नहीं
    \end{enumerate}

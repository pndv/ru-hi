\chapter{कारक}\label{ch:cases}

हिन्दी मे कारक कि तरह रूसी भाषा में भी कारक होते हैं। इन्हे संज्ञा, सर्वनाम, विशेषण इत्यादि में प्रयोग करते हैं।


\section{कर्ता कारक} \label{sec:nominative-case}


\section{कर्म कारक} \label{sec:accusative-case}


\section{करण कारक} \label{sec:instrumental-case}


\section{संबंध कारक}  \label{sec:prepositional-case}


\section{संप्रदान कारक}  \label{sec:dative-case}


\section{अधिकरण कारक} \label{sec:genitive-case}


\section{अपादान कारक} \label{sec:ablative-case}